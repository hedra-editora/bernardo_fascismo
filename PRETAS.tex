\textbf{Cleonildo Cruz} é historiador, cineasta e doutorando em Epistemologia e História da Ciência pela Universidad Nacional Tres de Febrero, Buenos Aires – Argentina. Filmografia: \emph{Replicar dos Sinos (expulsão do Pe. Vito Miracapillo do Brasil)}, 2005; \emph{Pernambuco: o golpe: 1964-1979}, 2008; \emph{Caixa de Pandora}, 2010; \emph{Haiti, 12 de janeiro}, 2012; \emph{Constituinte 1987-1988}; \emph{Operação Condor, verdade inconclusa}, 2015/16; e \emph{Olhares Anistia}, 2017. Publicou também o livro \emph{Constituinte 1987-1998}, pela editora da \versal{CFOAB} (Conselho Federal da Ordem dos Advogados do Brasil).

\textbf{Liana Cirne} é advogada. Doutora em Direito Público, mestra em Instituições Jurídicas-Políticas e professora da Faculdade de Direito da \versal{UFPE} (Universidade Federal de Pernambuco). Mãe e feminista. É colunista da Revista Fórum e do Jornal \versal{GGN}.

\textbf{A Constituição traída: Da abertura democrática ao golpe e à prisão de Lula}, sob meticulosa organização de Cleonildo Cruz e Liana Cirne, é um profundo e multifacetado exercício de reflexão sobre a República que se forma sob a égide da Constituição de 1988, em seus avanços, mas também em suas carências e reminiscências do autoritarismo que levaram ao atual estado de crise democrática. São reunidas formas e conteúdos literários dos mais diversos, que possibilitam uma abordagem ampla e completa do tema: três dos mais marcantes discursos políticos desde a redemocratização; uma série de ensaios de juristas, economistas, dentre outros; e diversas entrevistas com personagens ativos no processo político da Constituinte e do cenário político brasileiro.

