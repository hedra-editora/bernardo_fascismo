To_Do:

++ Nomes do meses em c.b., como padrão no Brasil?

++ Manter hifenização segundo padrão de Portugal (p.e., anarco-sindicalismo em vez de anarcossindicalismo)? --- SIM

++ Manter todas as letras dos título (livros, artigos, etc.) em caixa-alta?

++ Manter mão-de-obra hifenizado? --- SIM

++ manter preposição depois de até? --- SIM

++ Padronizar: op. cit. seguido de número de página, vírgula ou não após numeração?


+++ Padronizar: nacional-socialistas X nacionais-socialistas --- OK: Padrão português é nacionais-socialistas para substantivo e nacional-socialistas para adjetivo

++ Padronizar: \% ou por cento

++ não-judeu (judaico) X não judeu (judaico)?

verificar itálicos de epigrafe da Parte 6: Metamorfoses do fascismo

Questões estruturais:

++ É preciso tirar "Parte X", "Capítulo X" dos títulos, conforme padrão editorial, mas todas as referências que o autor faz ao próprio texto vem como "consultar capítulo X da Parte Y". Isso vai gerar confusão no leitor? Talvez manter, nos títulos, apenas a numeração do capítulo e/ou parte, sem o nome "capítulo" e "parte"?

++ Pensar na possibilidade de inserir \pageref indicando as páginas das notas às quais o autor faz referências ao longo do livro 

++ Quebra da página 492

